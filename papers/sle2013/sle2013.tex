\documentclass[runningheads,a4paper]{llncs}

\usepackage{amssymb}
\setcounter{tocdepth}{3}
\usepackage{graphicx}

\usepackage[utf8]{inputenc}
\usepackage[unicode=true,hidelinks]{hyperref}

\usepackage{url}
\newcommand{\keywords}[1]{\par\addvspace\baselineskip
\noindent\keywordname\enspace\ignorespaces#1}


\newcommand{\ie}{\emph{i.e.}}
\newcommand{\eg}{\emph{e.g.}}
\newcommand{\cf}{\emph{cf.~}}
\newcommand{\etal}{\emph{et al.~}}
\newcommand{\etc}{\emph{etc.}}
\newcommand{\aka}{\emph{a.k.a.}}


\begin{document}

\mainmatter

\title{FFI Is Not Enough. Need Dependent Types}
\titlerunning{Dependent Types Rock}

\author{Julien Richard-Foy \and Olivier Barais\and Jean-Marc Jézéquel}
\authorrunning{Julien Richard-Foy \emph{et. al.}}

\institute{IRISA, Université de Rennes 1, France. \texttt{\{first\}.\{last\}@irisa.fr}}

%\toctitle{Lecture Notes in Computer Science}
%\tocauthor{Authors' Instructions}
\maketitle


\begin{abstract}
The abstract should summarize the contents of the paper and should
contain at least 70 and at most 150 words. It should be written using the
\emph{abstract} environment.
\keywords{We would like to encourage you to list your keywords within
the abstract section}
\end{abstract}


\section{Introduction}

Web applications are attractive because they require no installation or deployment step on clients and enable large
scale collaborative experiences. However, writing large Web applications is known to be
difficult~\cite{Mikkonen08_SpaghettiJs,Preciado05_RIAMethodologyNecessity}. One challenge comes from the fact that
the JavaScript programming language -- which is currently the only action language natively supported by almost all
Web clients -- lacks of constructs making large code bases maintainable (\eg static typing, first-class modules).

One solution consists in considering JavaScript as an assembly language and generating JavaScript from compilers
of full-featured and cutting-edge programming languages. Incidentally, an increasing number of programming languages
or compiler backends can generate JavaScript code (\eg Java/GWT~\cite{Chaganti07_GWT},
SharpKit\footnote{\href{http://sharpkit.net}{http://sharpkit.net}}, Dart~\cite{Griffith11_Dart},
Kotlin\footnote{\href{http://kotlin.jetbrains.org/}{http://kotlin.jetbrains.org/}},
ClojureScript~\cite{McGranaghan11_ClojureScript}, Fay\footnote{\href{http://fay-lang.org/}{http://fay-lang.org/}},
Haxe~\cite{Cannasse08_HaXe}, Opa\footnote{\href{http://opalang.org/}{http://opalang.org/}}).

However, compiling to JavaScript is not enough. Developers also need the Web browser programming environment: they
need to interact with the Web page, to build DOM fragments, to listen to user events, \etc. A Foreign Function
Interface mechanism could be used to make browser’s APIs available to the developers. However, JavaScript APIs are
not statically typed and make a heavy use of overloading, making them hard to expose in a statically typed language.
Indeed, most of the existing statically typed languages compiling to JavaScript either lose control or type safety
when they expose Web browser’s APIs. How to give developers the same level of control as if they were using the
native Web APIs, but in a statically typed and convenient way?

In this paper we present several ways to integrate Web browser’s APIs as statically typed APIs that are safe and
give developers the same control level as if they were using the native APIs. We could achieve this
by using advanced features of type systems like dependent types and functional dependencies.

\section{Motivating Example}

Typical tasks involved in Web applications.

Why is it difficult to type Web browser’s APIs?

\section{Lightweight Modular Staging}

\section{Contribution}

\subsection{Events}

Path-dependent types to abstract over an event name and its data type.

\subsection{Selectors}

\begin{itemize}
 \item Less type annotations on DOM queries, less chance to write nonsense casts
 \item Inference-driving macros help inferring more specific types
\end{itemize}

\subsection{DOM}

?

\section{Evaluation}

\subsection{Events}

Other languages either provide lose information about the data type of the listened event (Dart) or give no way to
abstract over an event (GWT, Kotlin).

\section{Conclusion and Perspectives}

\begin{thebibliography}{4}

\bibitem{jour} Smith, T.F., Waterman, M.S.: Identification of Common Molecular
Subsequences. J. Mol. Biol. 147, 195--197 (1981)

\bibitem{lncschap} May, P., Ehrlich, H.C., Steinke, T.: ZIB Structure Prediction Pipeline:
Composing a Complex Biological Workflow through Web Services. In: Nagel,
W.E., Walter, W.V., Lehner, W. (eds.) Euro-Par 2006. LNCS, vol. 4128,
pp. 1148--1158. Springer, Heidelberg (2006)

\bibitem{book} Foster, I., Kesselman, C.: The Grid: Blueprint for a New Computing
Infrastructure. Morgan Kaufmann, San Francisco (1999)

\bibitem{proceeding1} Czajkowski, K., Fitzgerald, S., Foster, I., Kesselman, C.: Grid
Information Services for Distributed Resource Sharing. In: 10th IEEE
International Symposium on High Performance Distributed Computing, pp.
181--184. IEEE Press, New York (2001)

\bibitem{proceeding2} Foster, I., Kesselman, C., Nick, J., Tuecke, S.: The Physiology of the
Grid: an Open Grid Services Architecture for Distributed Systems
Integration. Technical report, Global Grid Forum (2002)

\bibitem{url} National Center for Biotechnology Information, \url{http://www.ncbi.nlm.nih.gov}

\end{thebibliography}

\end{document}
